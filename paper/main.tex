\documentclass[thesismargins, english, thesislinespacing, twoside, openright, upjsfrontpage]{rnthesis}
\usepackage[english]{babel}
\usepackage[T1]{fontenc}
\usepackage[utf8]{inputenc}
\usepackage{lmodern}

\usepackage{rnt-pic}
\usepackage{rnt-thm}

\usepackage{pdfpages} % inserting pdf pages

\usepackage[hyphens]{url} % format and linebreak of URLs

\title{Open electronic signature software}
\author{Jakub Ďuraš}
\typprace{bachelor's}
\rok{2020}
\odbor{Applied Informatics}
\miesto{Košice}
\podakovanie{
  Thanks to the Planet!
}
\veduci{RNDr. Viliam Kačala}
\pracovisko{Institute of Computer Science}
\pdfzadanie{zadanie.pdf}

\abstract{
With the recent changes in the legal status of electronic signatures in many parts of the world, there is a need for easily accessible solutions intended as an alternative to handwritten signatures. This may be more necessary than ever since electronic communication is the preferred way of communication. This bachelor thesis aims to explore the principles and legal status of electronic signatures, review current electronic signature software, and explore possible obstacles the open-source community is facing to develop such specialized applications. We propose an open-source, cross-platform, and user-friendly software compliant with the eIDAS Regulation (Regulation No 910/2014). Our application should allow ordinary users to quickly sign and verify signatures of different types of documents and therefore easily use them in everyday life.
}

\keywords{electronic signature, electronic seal, qualified, open-source, XAdES, PAdES, CAdES, desktop software}

\abstrakt{
Abstrakt v SK jazyku bude pridaný neskôr ako voľný preklad z EN verzie (pozri ďalšiu stranu). Predíde sa tak zbytočnej práci keďže EN abstrakt sa môže časom meniť.
}

\klucoveslova{elektronický podpis, elektronická pečať, kvalifikovaný, open-source, XAdES, PAdES, CAdES, počítačový softvér}

\bibliographystyle{alpha}

\begin{document}
\maketitle
\newpage
\tableofcontents
\listoffigures
\listoftables
% zoznam značiek a skratiek. Pre tento koncept neexistuje
% LaTeXovsky príkaz

\uvod

Lorem ipsum dolor sit amet, consectetuer adipiscing elit.
Integer lacinia, nulla porta varius tempus, lacus metus blandit
lorem, a rutrum justo wisi id sapien. Integer risus libero,
feugiat eleifend, ornare ac, volutpat nec, sem. In facilisis,
quam eu elementum aliquet, lorem quam euismod dui, aliquet
laoreet purus ipsum ac quam.


\chapter{Background and theory}

In our work, we rely on knowledge in several different subjects that we go over in the following sections.
Firstly, the current state of the law and recent changes relating to the use of electronic signatures.
Secondly, on the well-established concepts in cryptography like asymmetric cryptography or hashing.
Lastly, expertise from the field of software engineering - especially design, implementation, and maintenance of computer software.
All in relation to the electronic signatures, and, where applicable, open-source and desktop computer software.

\section{Law}

Unless specified otherwise, we are considering law applicable locally in the Slovak Republic (SR).
Its law is greatly influenced by the European Union (being its member since 2004) and to some extent by the rest of the world.
Therefore, we can assume this is at least partially applicable outside of the SR.

\subsection{Signatures}

Signatures are essential part of the written legally binding documents like contracts.
They are permanently affixed to the document and are supposed to uniquely identify the person and its deliberate, informed consent.
As can be seen in the Slovak Civil Code, "A written legal act is valid if signed by the acting person;"\cite{1}.
Law often explicitly requires signatures and further clarifies their expected use.
For example, when selling an enterprise, "The contract requires written form and attested signatures of the seller and the buyer."\cite{2}.

Signature is considered an "attested signature"\footnote{Translation from "osvedčený podpis" as used in the Slovak law.} if it is attested by the authorized third party.
Such process, known as legalization, is regulated by the law and its purpose is to attest information which could form the basis for the exercise of rights or which could cause legal consequences\cite{3}.

Historically, \todo{Comment on law and electronic signatures}

\subsection{European Union Regulation eIDAS}

With intention to stimulate digital growth, the European Union established regulation on electronic identification and trust services for electronic transactions in the European Single Market (eIDAS).
It applies from 1st of July 2016 and, among other things, it regulates electronic signatures focusing on the:

1. Electronic signatures in general.
2. Qualified electronic signatures.
3. Trust service.

\todo{Explain and cite how electronic signatures can be used instead of hand-written one and qualified instead of attested}

\subsection{Copyright and computer software}

Lorem ipsum dolor sit amet, consectetuer adipiscing elit.

\section{Cryptography}

- 

\section{Software engineering}

- 

\chapter{Methodology}

Lorem ipsum dolor sit amet, consectetuer adipiscing elit.
Integer lacinia, nulla porta varius tempus, lacus metus blandit
lorem, a rutrum justo wisi id sapien. Integer risus libero,
feugiat eleifend, ornare ac, volutpat nec, sem. In facilisis,
quam eu elementum aliquet, lorem quam euismod dui, aliquet
laoreet purus ipsum ac quam.

\chapter{Suggested solution}

Lorem ipsum dolor sit amet, consectetuer adipiscing elit.
Integer lacinia, nulla porta varius tempus, lacus metus blandit
lorem, a rutrum justo wisi id sapien. Integer risus libero,
feugiat eleifend, ornare ac, volutpat nec, sem. In facilisis,
quam eu elementum aliquet, lorem quam euismod dui, aliquet
laoreet purus ipsum ac quam.

\chapter{Results}

Lorem ipsum dolor sit amet, consectetuer adipiscing elit.
Integer lacinia, nulla porta varius tempus, lacus metus blandit
lorem, a rutrum justo wisi id sapien. Integer risus libero,
feugiat eleifend, ornare ac, volutpat nec, sem. In facilisis,
quam eu elementum aliquet, lorem quam euismod dui, aliquet
laoreet purus ipsum ac quam.

Donec dolor arcu, posuere at, vehicula vitae, accumsan ut,
lacus. Nulla tristique eros eu diam. Vivamus nec tortor vel
ligula elementum lacinia. Curabitur euismod eros adipiscing
ipsum. Donec sed quam at felis suscipit egestas. Morbi faucibus
libero sit amet libero. Nullam laoreet ipsum eu eros. Donec in
diam. Ut facilisis eros vel leo. Nunc vitae mauris. Donec leo
erat, luctus porttitor, laoreet eget, facilisis non, erat.
Integer nec elit.

\zaver

Ut lobortis semper risus, non condimentum dui convallis ut. Nulla eget volutpat tellus. Vestibulum lobortis tincidunt massa eu rhoncus. Suspendisse luctus eu dui non vehicula. Vivamus elementum auctor felis, placerat maximus magna lobortis ut. Donec placerat sem a mi sagittis blandit. Maecenas pellentesque laoreet mauris, dictum viverra ante sodales sed. Nullam non ligula quis ante ultricies finibus non quis ex. Ut tempor vitae ipsum sed imperdiet. Fusce aliquam nisl sit amet nunc tempor convallis. Vivamus vehicula magna sit amet purus commodo, et tincidunt purus accumsan. Nulla velit dolor, lacinia nec scelerisque a, euismod a sapien.
%

%\renewcommand{\bibname}{Zoznam použitej literatúry}
\begin{thebibliography}{9}
  % Príklady popisu dokumentov citácií podľa systému meno a dátum (Harvardský systém)
  % ----
  % Varianty zápisov autorov:
  % [1] GUZANIN, Štefan, Robert SABOVČÍK a Pavol KAČMÁR. Priezviská vždy VEĽKÝMI PÍSMENAMI,
  %   priezvisko prvého autora je vždy pred menom, druhý a ďalší autor majú zápis
  %   Meno PRIEZVISKO
  % [2] Neuvádzať rodné mená autorov.
  % [3] Verzálky nie sú povinné, možno použiť aj iné indikatívnejšie označenie
  %
  % --- 
  % 1. Knižná publikácia (monografia, učebnica, zborník ...)
  %   1 autor

  \bibitem{1}
  \emph{§ 40 ods. 3 zákona č. 40/1964 Zb. občiansky zákonník}

  \bibitem{2}
  \emph{§ 476 ods. 2 zákona č. 513/1991 Zb. obchodný zákonník}

  \bibitem{3}
  \emph{§ 56 ods. 1 zákona č. 323/1992 Zb. notársky poriadok}

  %\bibitem{2}
  %\osoba{Beck, G.}, 2007. \emph{Zakázaná rétorika: 30 manipulatívních technik}. Preklad
  %\osoba{Pomikálová, M.}. Praha: Grada Publishing. ISBN 978-80-247-1743-2.
  %\bibitem{3}
  %\osoba{Vojčík, P.}, 2010. \emph{Občianske právo hmotné II.} 3. prep. a dopl. vyd. Košice: UPJŠ v Košiciach. ISBN 978-80-7097-817-7.
  %  2 autori
  %\bibitem{4}
  %\osoba{Šoltés, M.} a \osoba{Radoňák, J.}, 2013. \emph{Základné princípy laparoskopickej chirurgie.} Košice: UPJŠ v Košiciach. ISBN 978-80-8152-074-7.
  % 3 autori
  %\bibitem{5}
  %\osoba{Guzanin, Š.}, \osoba{Sabovčík, R.} a \osoba{Kačmár, P.}, 2004. \emph{Selected Chapters of Plastic and Reconstructive Surgery: vysokoškolské učebné texty}. Košice: Univerzita Pavla Jozefa Šafárika v Košiciach, Lekárska fakulta. ISBN 80-7097-557-1.
  % 4 a viac autorov
  %\bibitem{6}
  %\osoba{Nagyová, I.} et al. 2009. \emph{Measuring health and quality of life in the chronically ill}. Košice: Equilibria. ISBN 978-80-892-8446-7.
  % Elektronická kniha
  %\bibitem{7}
  %\osoba{Speight, J. G.}, 2005. \emph{Lange's Handbook of Chemistry} [online]. London: McGraw-Hill. [cit. 2009.06.10.] ISBN 978-1-60119-261-5. Dostupné na: \url{http://www.knovel.com/web/portal/basic_search/display?_EXT_KNOVEL_DISPLAY_bookid=1347&_EXT_KNOVEL_DISPLAY_fromSearch=true&_EXT_KNOVEL_DISPLAY_searchType=basic}
  % zborník
  %\bibitem{8}
  %\osoba{Bačkor, M.} a \osoba{Mihaličová, S.}, zost., 2013. \emph{Zborník príspevkov z konferencie 11. dni doktorandov experimentálnej biológie rastlín a 13. konferencie experimentálnej biológie rastlín} [online]. Košice: Univerzita Pavla Jozefa Šafárika v Košiciach, Prírodovedecká fakulta [cit. 2009-06-10]. ISBN 9788081520327. Dostupné na: \url{
  %  http://www.upjs.sk/public/media/5596/PF-Zbornik-prispevkov-konferencie-11-dni-doktorandov.pdf}
  %
  % 2. Časopis (ako celok)
  %\bibitem{9}
  %\emph{Thaiszia: Journal of Botany}. Košice: P.J.Safarik University, Botanic Garden, \mbox{1990--\ .} ISSN 1210-0420.
  %\bibitem{10}
  %\emph{Ikaros: elektronický časopis o informační bezpečnosti} [online], 2002. [Praha]: Ikaros. 1997--{} [cit. 2002-03-08]. Dostupné na: \url{http://www.ikaros.cz/}. ISSN 1212-5075.
  % Jedno číslo časopisu
  %\bibitem{11}
  %\emph{CHIP: magazín informačních technologií}, 2013. Praha: Burda Praha, roč. 23, říjen. ISSN 1210-0684.
  %  ------
  %  3. Príspevok v knihe/zborníku
  %  ------
  %\bibitem{12}
  %\osoba{Sabol, J.}, 2000. Jazyk ako ľudské posolstvo: (namiesto doslovu). In: \emph{O jazyku a štýle kriticky aj prakticky}. Prešov: Náuka, s. 149--159. ISBN 809676022X.
  %\bibitem{13}
  %\osoba{Tóthová, E.} a kol., 2013. A rare t(9,22,16)(q34,q11,q24) translocation in chronic myeloid leukemia for which imatinib mesylate was effective: a case report. In: \emph{XXVII. Olomoucké hematologické dny s mezinárodní účastí, 12.--14.5.2013, Olomouc: sborník abstrakt}. Olomouc: Univerzita Palackého v Olomouci, s. 75--76. ISBN 9788024434803.
  %  ------
  % 4. Článok v časopise
  %  ------
  %\bibitem{14}
  %\osoba{Beňačka, J.} et al., 2009. A better cosine approximate solution to pendulum equation. In: \emph{International Journal of Mathematical Education in Science and Technology}. Vol. 40, no. 2, p. 206--215. ISSN 0020-739X.
  %\bibitem{15}
  %\osoba{Dubayová, T.} et al., 2010. The impact of the intensity of fear on patient's delay regarding health care seeking behavior: a systematic review vyhľadaní zdravotníckej starostlivosti. In: \emph{International Journal of Public Health}. Vol. 55, no. 5, p. 459--468. ISSN 1661-8556.
  %\bibitem{16}
  %\osoba{Steinerová, J.}, 2000. Princípy formovania vzdelania v informačnej vede. In: \emph{Pedagogická revue}. Roč. 2, č. 3, s. 8--16. ISSN 1335-1982.
  %\bibitem{17}
  %\osoba{Hoggan, D.}, 2002. Challenges, Strategies, and Tools for Research Scientists. In: \emph{Electronic Journal of Academic and Special Librarianship} [online]. Vol. 3, no. 3 [cit. 2013-01-10]. ISSN 1525-321X. Dostupné na: \url{http://southernlibrarianship.icaap.org/content/v03n03/Hoggan_d01.htm}
  %\bibitem{18}
  %\osoba{Srbecká, Gabriela}, 2010. Rozvoj kompetencí studentů ve vzdělávání. In: \emph{Inflow: information journal} [online]. Roč. 3, č. 7 [cit. 2013-08-06]. ISSN 1802-9736. Dostupné na: \url{http://www.inflow.cz/rozvoj-kompetenci-studentu-ve-vzdelavani}
  %  ------
  % 5. Príspevok v zborníku na CD-ROM
  %  ------
  %\bibitem{19}
  %\osoba{Zemánek, P.}, 2001. The machines for ``green works'' in vineyards and their economical evaluation. In: \emph{9th International Conference: proceedings. Vol. 2. Fruit Growing and viticulture} [CD-ROM]. Lednice: Mendel University of Agriculture and Forestry, p. 262--268. ISBN 80-7157-524-0.
  %  ------
  % 6. Záverečné a kvalifikačné práce
  %  ------
  %\bibitem{20}
  %\osoba{Mikulášiková, M.}, 1999. \emph{Didaktické pomôcky pre praktickú výučbu na hodinách výtvarnej výchovy pre 2. stupeň základných škôl}: diplomová práca. Nitra: UKF.
  %\bibitem{21}
  %\osoba{Urdzík, P.}, 2007. \emph{Predikcia intrauterinnej rastovej retardácie a preeklampsie pomocou biochemických a ultrazvukových markerov}: dizertačná práca. Košice: UPJŠ v Košiciach.
  %  ------
  % 7. Výskumné správy
  %  ------
  %\bibitem{22}
  %\osoba{Baumgartner, J.} a kol., 1998. \emph{Ochrana a udržiavanie genofondu zvierat, šľachtenie zvierat}: výskumná správa. Nitra: VÚŽV.
  %  ------
  % 8. Normy
  %  ------
  %\bibitem{23}
  %STN ISO 690: 2012. \emph{Informácie a dokumentácia. Návod na tvorbu bibliografických odkazov na informačné pramene a ich citovanie}.
  %  ------
  % 9. Mapa
  %  ------
  %\bibitem{24}
  %VKÚ, 2003. \emph{Košice: mapa okolia}. [1:15000]. 3. vyd. Harmanec: VKÚ. ISBN 80-8042-223-0.
  %  ------
  % 10. Zákon
  %  ------
  %\bibitem{25}
  %\emph{Zákon č.131/2002 Zb. o vysokých školách a o zmene a doplnení niektorých zákonov.}
  %\bibitem{26}
  %\emph{Zákon č. 313/2001 o verejnej službe.}

\end{thebibliography}

\end{document}
